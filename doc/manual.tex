
\section{Vernacular Commands}

\subsection{Derive}

\Equations comes with a suite of deriving commands that take inductive
families and generate definitions based on them. The common syntax for
these is:

\[\mathtt{Derive}~\ind{C}_1 \ldots \ind{C}_n~\mathtt{for}~\ind{ind}_1 \ldots \ind{ind}_n.\]

Which will try to generate an instance of type class \ind{C} on
inductive type \ind{Ind}. We assume $\ind{ind}_i : Π Δ. s$.
The derivations provided by \Equations are:

\begin{itemize}
\item \ind{DependentEliminationPackage}
\item \ind{Signature}
\item \ind{NoConfusion}
\item \ind{Equality}. 
  This derives a decidable equality on $C$, assuming decidable equality 
  instances for the parameters and supposing any primitive inductive
  type used in the definition also has decidable equality. If
  successful it generates an instance of the class:
\begin{verbatim}
Class EqDec (A : Type) :=
  eq_dec : forall x y : A, { x = y } + { x <> y }.
\end{verbatim}
  
\item \ind{Subterm}
\end{itemize}



%%% Local Variables: 
%%% mode: latex
%%% TeX-master: "equations"
%%% TeX-PDF-mode: t
%%% End: 
