\subsection{Recursion}
\label{sec:recursion}

We will now explain how we treat recursive definitions by
pattern-matching. A notorious problem with the \Coq system is that it
uses a syntactic check to verify that recursive calls are well-formed. 
Only structurally recursive functions making recursive calls on a
designated argument of a recursive function are allowed.
The syntactic criterion is very restrictive and is a major source of
bugs in the core type checker. A typical problem arises when one 
wants to reabstract recursive subterms to change their type.
Consider the following slighly contrived definition for example:

\coqdockw{Fixpoint} \coqdef{recfail.filter}{filter}{\coqdocdefinition{filter}} \{\coqdocvar{A} : \coqdockw{Type}\} (\coqdocvar{p} : \coqdocvariable{A} \ensuremath{\rightarrow} \coqexternalref{http://coq.inria.fr/distrib/trunk/stdlib/Coq.Init.Datatypes}{bool}{\coqdocinductive{bool}}) (\coqdocvar{l} : \coqexternalref{http://coq.inria.fr/distrib/trunk/stdlib/Coq.Init.Datatypes}{list}{\coqdocinductive{list}} \coqdocvariable{A}) \{\coqdockw{struct} \coqdocvar{l}\} :=\coqdoceol
\coqdocindent{1.00em}
\coqdockw{match} \coqdocvariable{l} \coqdockw{with}\coqdoceol
\coqdocindent{2.00em}
\ensuremath{|} \coqexternalref{http://coq.inria.fr/distrib/trunk/stdlib/Coq.Init.Datatypes}{nil}{\coqdocconstructor{nil}} \ensuremath{\Rightarrow} \coqexternalref{http://coq.inria.fr/distrib/trunk/stdlib/Coq.Init.Datatypes}{nil}{\coqdocconstructor{nil}}\coqdoceol
\coqdocindent{2.00em}
\ensuremath{|} \coqexternalref{http://coq.inria.fr/distrib/trunk/stdlib/Coq.Init.Datatypes}{cons}{\coqdocconstructor{cons}} \coqdocvar{a} \coqdocvar{l} \ensuremath{\Rightarrow} \coqdoceol
\coqdocindent{3.00em}
\coqdockw{match} \coqdocvariable{p} \coqdocvariable{a} \coqdockw{return} \coqexternalref{http://coq.inria.fr/distrib/trunk/stdlib/Coq.Init.Datatypes}{list}{\coqdocinductive{list}} \coqdocvariable{A} \ensuremath{\rightarrow} \coqexternalref{http://coq.inria.fr/distrib/trunk/stdlib/Coq.Init.Datatypes}{list}{\coqdocinductive{list}} \coqdocvariable{A} \coqdockw{with}\coqdoceol
\coqdocindent{4.00em}
\ensuremath{|} \coqexternalref{http://coq.inria.fr/distrib/trunk/stdlib/Coq.Init.Datatypes}{true}{\coqdocconstructor{true}} \ensuremath{\Rightarrow} \coqdockw{fun} \coqdocvar{l} \ensuremath{\Rightarrow} \coqdocvariable{a} :: \coqref{recfail.filter}{\coqdocdefinition{filter}} \coqdocvariable{p} \coqdocvariable{l}\coqdoceol
\coqdocindent{4.00em}
\ensuremath{|} \coqexternalref{http://coq.inria.fr/distrib/trunk/stdlib/Coq.Init.Datatypes}{false}{\coqdocconstructor{false}} \ensuremath{\Rightarrow} \coqdockw{fun} \coqdocvar{l} \ensuremath{\Rightarrow} \coqref{recfail.filter}{\coqdocdefinition{filter}} \coqdocvariable{p} \coqdocvariable{l}\coqdoceol
\coqdocindent{3.00em}
\coqdockw{end} \coqdocvariable{l}\coqdoceol
\coqdocindent{1.00em}
\coqdockw{end}.\coqdoceol

Here we reabstract on the \var{l} variable in the inner match, as would
be done by the compilation of a with node for example. This seemingly 
innocuous change is too much for the guardness criterion, it cannot
check that the \var{l} variable on which we make the recursive call
is actually always substituted by the subterm \var{l} applied to
the \kw{match} expression. Hence our program manipulations preclude
the use of syntactically guarded recursion. However we 
can use the same principle as for pattern-matching and 
\emph{explain} the recursive structure of our programs using type theory
itself. 

To do so, we will use an elimination principle on the datatype we want
to recurse on, that will give us a way to make recursive calls on any
subterm. Instead of a syntactic notion of structural recursion, we will now use a
logical one, which is compatible with the rest of the logical
transformations happening during compilation, like the above use of
strenghtening.

\subsubsection{The \Below way}

\cite{DBLP:conf/birthday/GoguenMM06} describes a clever way to
elaborate recursive definitions by building a memoizing structure. 
For each datatype, we define a new type \ind{Below} that describes all
the recursive subterm of a given term. Take natural numbers for example,
we define \cst{Below\_nat} as follows:

\input{Below.coq}

This method handles the definition structurally recursive definitions
satisfactorily, but it is very inneficient. Indeed, if we try to reduce 
a program using this recursor, there will be an exponential blowup as 
the object we are recursing on, that is the tuple of all possible
recursive calls \cst{below\_nat}, will have to be computed for each call. This is not 
a problem if we are using a lazy reduction strategy but it is
prohibitive if we want to compute with a call-by-value strategy, 
or compute with the extracted program in \ML. Extraction removes the 
logical parts of a term (in \Prop), like the manipulations on equality
used during specialization by unification, but in this case the
\cst{Below} object is computational and must be kept.

To avoid this problem, we will use another way of witnessing the subterm
relation that is entirely logical.

\subsubsection{Generalized subterm relations}

Our solution is basically to define the subterm relation on an inductive
family and write functions by well-founded recursion on this relation.

\begin{definition}[Subterm relation]
  Given an inductive type $\ind{I} : Π~Δ, τ$ with constructors 
  $\vec{\cstr{I}_i : Π~Γ_i, \ind{I}~Γ_i}$, we define the generalized 
  subterm relation as an inductive type $\ind{I}^{sub} : Π~Δ_l~Δ_r,
  \ind{I}~\bar{Δ_l} "->" \ind{I}~\bar{Δ_r} "->" \Prop$.
  For each constructor $\cstr{I}_i : Π~Γ_i, \ind{I}~Δ_i$ and for 
  each binding of $Γ_i$ of the form $(\var{x} : Π~Γ_x, \ind{I}~\vec{t})$
  we add a constructor to the relation: 
  $\cstr{\ind{I}}^{sub}_n : Π~Γ_i~Γ_x, 
  \ind{I}^{sub}\vec{t}~Δ_i~(\var{x}~\bar{Γ_x})~(\cstr{I}_i~\bar{Γ_i})$.
\end{definition}

Before going further, we will simplify our development by considering only
inductive families with a single index. Indeed we can define for any
inductive type $Π~Δ,~\ind{I}~Δ$ (any arity in general) a corresponding 
type with a single parameter using a dependent sum to encode the
context.

\def\teles#1{\ensuremath{Σ(#1)}}
\def\bars#1#2{\ensuremath{\bar{Σ}(#1, #2)}}
\begin{definition}[Telescope transformation]
  For any arbitrary context Δ, we define \teles{Δ} by recursion on the
  context:
  \begin{itemize}
    \item $\teles{ε} = \ind{unit}$
    \item $\teles{x : τ, Δ} = Σ x : τ, \teles{Δ}$
  \end{itemize}

  We omit type annotations for the construction of sums and the
  projections, they can be easily infered.
  We can define the $\bars{Δ}{s}$ operator on telescopes to produce
  appropriate projections from a tuple $s$:
  \begin{itemize}
  \item $\bars{ε}{s} = ε$
  \item $\bars{x : τ, Δ}{s} = \pi_1~s, \bars{Δ}{\pi_2~s}$
  \end{itemize}
\end{definition}

The heterogeneous subterm relation can hence be wrapped to form an
object of type:
\[Π i~i' : \teles{Δ}, \ind{I}~\bars{Δ}{i} "->" \ind{I}~\bars{Δ}{i'} "->" \Prop\]
%λ(i : \teles{Δ})~(i' :
%\teles{Δ}),\constr{I}^{sub}~\bars{Δ}{i}~\bars{Δ}{i'} :

\input{wf.coq}

We are almost ready to define structurally recursive
functions by well-founded recursion on the transitive closure of the
immediate subterm relation. The only remaining bit is to provide a
proof search procedure to automatically build proofs of the relation,
for automatically filling the witnesses that appear at recursive calls.
We can easily do so by building a hint database with the constructors of 
the $\ind{I}^{sub}$ relation and the constructors for the transitive closure
relation. 

\paragraph{Measures}

Using this method of recursion allows us to produce more efficient
programs but also opens up more flexibility. Indeed, we need not
restrict ourselves to using the subterm relation for building
well-founded definitions, we can also use any other available well-founded
relation at our hands. A common one is provided by the inverse image 
relation produced by a function on a given relation, often refered to
as a measure when the relation is the less-than order on natural
numbers.

%%% Local Variables: 
%%% mode: latex
%%% TeX-PDF-mode: t
%%% TeX-master: "equations"
%%% End: 