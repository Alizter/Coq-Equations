\documentclass{article}

\usepackage{me}
\usepackage{abbrevs}
\usepackage[color]{coqdoc}
\usepackage{coq}

\setlength{\coqdocbaseindent}{1.3em}

\def\name#1{\textsc{#1}~}
\def\Coq{\name{Coq}}
\def\Haskell{\name{Haskell}}
\def\text#1{#1}
\def\Ltac{$\mathcal{\text{L}}_{\text{tac}}$}
\def\Equations{\name{Equations}}
\def\coqlibrary#1#2#3{}

\author{Matthieu Sozeau}
\title{\Equations\\
  A dependent pattern-matching compiler}

\begin{document}
\maketitle

\begin{abstract}
  Equations is a compiler for definitions made by pattern matching on
  inductive families. It allows to write complex, recursive
  dependently-typed functions, automatically find a realization in \Coq 
  and generate proofs to ease reasoning on them.

  The basic functionality is the ability to define a function by a set
  of \textit{equations} with patterns on the left-hand side and programs
  in the right-hand side, in a similar fashion to \Haskell and \Agda
  function definitions. The system also supports with-clauses that can
  be used to add a pattern on the left-hand side for further refinement
  and also permits using arbitrary tactics to refine a programming
  problem. Both "syntactic" structural recursion and "semantic" well-founded 
  recursion schemes are usable in definitions, the later being
  generalized enough to cope with general inductive families.
  
  The system provides proofs of the equations that can be used as
  rewrite rules to simplify calls to the function in proofs. It also
  automatically generates the inductive graph of the function and a
  proof that the function respects it, giving a functional induction
  principle for it. 
  
  The whole system makes heavy use of type classes and \Ltac tactics to
  allow genericity and extensibility in a lightweight way.
\end{abstract}

\section{A gentle introduction to \Equations}

\input{intro}

  
\bibliographystyle{plainnat}
\bibliography{biblio}

\end{document}


%%% Local Variables: 
%%% mode: latex
%%% TeX-PDF-mode: t
%%% TeX-master: t
%%% End: 